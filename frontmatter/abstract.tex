%!TEX root = ../dissertation.tex
% the abstract

How should we attempt to understand the mind? Historically, there have been two broad approaches. The \emph{rational} approach focuses on characterizing the problems people have to solve and the optimal solutions to those problems, explaining \emph{why} people behave in the way they do. In contrast, the \emph{mechanistic} approach focuses on identifying the cognitive processes underlying behavior, explaining \emph{how} the mind actually works. Traditionally, these approaches have been viewed as conflicting, but recent years have seen a growing interest in models that synthesize the two approaches.

This dissertation presents a formal framework for deriving models of cognition that are both rational and mechanistic. The key idea to broaden the concept of the ``environment'' to which cognition adapts: cognitive processes are adapted not only to the external environment (the world), but also to the internal environment (the brain). Formalizing this old idea, I cast cognition as a sequential decision problem in which an agent executes cognitive actions to navigate between mental states and, ultimately, produce effective behavior. In three domains---attention, memory, and planning---I show how the framework can be applied to yield models that explain both how the mind works and why it works that way.


% How can we build theoretically satisfying and practically useful models of human cognition? Historically, there have been two broad approaches. The \emph{rational} approach focuses on characterizing the problems people have to solve and the optimal solutions to those problems. Rational models can explain \emph{why} people behave the way they do, and they can make generalizable predictions about how they will behave in different contexts. However, rational models do not explain how the mind produces that behavior, and they often fail to accurately predict how people actually behave. In contrast, the \emph{mechanistic} approach focuses on identifying the cognitive processes underlying behavior. Mechanistic models can potentially tell us \emph{how} the mind actually works, and they can explain why people don't always achieve the rational ideal. However, they do not explain why people use those cognitive processes, and they are often tuned to particular experimental contexts.

% Historically, these approaches have viewed as conflicting, but recent years have seen a growing interest in models that synthesize the two approaches.

% How can we build theoretically satisfying and practically useful models of human cognition? Historically, there have been two broad approaches. The \emph{rational} approach focuses on characterizing the problems people have to solve and the optimal solutions to those problems. Rational models can explain \emph{why} people behave the way they do, and they can make generalizable predictions about how they will behave in different contexts. However, rational models do not explain how the mind produces that behavior, and they often fail to accurately predict how people actually behave. In contrast, the \emph{mechanistic} approach focuses on identifying the cognitive processes underlying behavior. Mechanistic models can potentially tell us \emph{how} the mind actually works, and they can explain why people don't always achieve the rational ideal. However, they do not explain why people use those cognitive processes, and they are often tuned to particular experimental contexts.
