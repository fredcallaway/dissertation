%!TEX root = ../dissertation.tex
% the acknowledgments section

\newthought{Being a scientist} is, I think, one of the greatest privileges a human being can experience. I am enormously grateful to all the people who have made it possible for me to experience it.

I thank Paul Pease, my highschool biology teacher, for planting in my mind the seed of scientific curiosity.

I thank Shimon Edelman, my undergraduate advisor, for nurturing the budding plant---for taking my na\"ive and ill-conceived ideas seriously, for instilling in me an appreciation for both function and process, and for demonstrating that one person can be both an engineer and an artist.

And I thank Tom Griffiths, my graduate advisor, for teaching me how to garden---for showing more than telling, for pushing me to do things the right way rather than the easy way, for surrounding me with kind and brilliant collaborators, and, most of all, for showing me not just how to do good science, but how to be a good scientist.

Speaking of those kind and brilliant collaborators, I thank all the co-authors I have had the pleasure of working with in graduate school: Antonio Rangel, Bas van Opheusden, Carlos Correa, Emily Liquin, Erin Grant, Falk Lieder, Gustav Karreskog, James Hillis, Jess Hamrick, Ken Norman, Mark Ho, Matt Hardy, Paul Krueger, Priyam Das, Qiong Zhang, Sayan Gul, Tania Lombrozo, Vael Gates, and Yash Jain.

In particular, I would like to thank the graduate students and postdocs who acted as mentors to me (in chronological order): Jess Hamrick, Falk Lieder, and Bas van Opheusden. In each of them I have found both invaluable technical skills and also a role model for the type of scientist I would like to be: rigorous, passionate, patient, knowledgeable, and kind.

I also thank Antonio Rangel, for the incisiveness he brought to the work in Chapter~\ref{sec:attention}, which I have tried to emulate in all my work since.

I thank the organizers of the SLOAN-Nomis workshop on the cognitive foundations of economic behavior---Andrew Caplin, Ernst Fehr, and Michael Woodford---for organizing the most stimulating academic events I have had the privilege of attending, where I first met Antonio, and where I recently escaped from a prolonged lapse in enthusiasm for science.

I thank the Julia development team for designing an efficient high-level programming language, without which I would not not have been able to do the work in this dissertation.

I thank the Princeton Psychology/Neuroscience IT staff for maintaining the hardware necessary for me to run that Julia code, in particular, John Wiggins and Garrett McGrath.

I thank the administrative staff of the Princeton Psychology department for creating a welcoming and supportive environment, and in particular Jill Ray for guiding me through the graduation process.

I thank the members of my disseration committee---Jon Cohen, Ken Norman, Marcelo Mattar, Nathaniel Daw, and Tom Griffiths---for their comments and guidance as I wrote this dissertation.

I thank my mother and father, for prioritizing my education, for supporting the things that made me a better and happier person (and discouraging the rest),  and for fostering in me the sense that my achievements would be bounded only by what I wished to achieve.\footnote{%
  Unfortunately, this turns out not to be true, but it's a nice thing to believe for the first twenty years or so.
}












