%!TEX root = ../dissertation.tex
\begin{savequote}[75mm]
Suppose we try to recall a forgotten name. The state of our consciousness is peculiar. There is a gap therein; but no mere gap. It is a gap that is intensely active. A sort of wraith of the name is in it, beckoning us in a given direction, making us at moments tingle with the sense of our closeness, and then letting us sink back without the longed-for term.
\qauthor{William James \citeyear{james1890principles}}
\end{savequote}

% https://www.brainyquote.com/quotes/marcel_proust_401575/
% https://www.brainyquote.com/quotes/linus_torvalds_381583

% “It’s not that I can’t remember. It’s that I prefer not to remember, which means that I prefer not to remember what not remembering did to me the last time I did it.” ― Craig D. Lounsbrough, An Autumn's Journey: Deep Growth in the Grief and Loss of Life's Seasons

% “Your memory is a monster; you forget—it doesn't. It simply files things away. It keeps things for you, or hides things from you—and summons them to your recall with will of its own. You think you have a memory; but it has you!” ― John Irving, A Prayer for Owen Meany

% https://www.goodreads.com/quotes/tag/memory?page=4

\Chapter[]{Memory}\label{sec:memory}


